% Created 2019-03-01 Fri 12:00
% Intended LaTeX compiler: pdflatex
\documentclass{article}
\usepackage[utf8]{inputenc}
\usepackage[T1]{fontenc}
\usepackage{graphicx}
\usepackage{grffile}
\usepackage{longtable}
\usepackage{wrapfig}
\usepackage{rotating}
\usepackage[normalem]{ulem}
\usepackage{amsmath}
\usepackage{textcomp}
\usepackage{amssymb}
\usepackage{capt-of}
\usepackage{hyperref}
\author{Daiwei Chen}
\date{\textit{<2019-02-27 Wed>}}
\title{Exam Review\\\medskip
\large D-don't Act Dumb}
\hypersetup{
 pdfauthor={Daiwei Chen},
 pdftitle={Exam Review},
 pdfkeywords={algorithms design review},
 pdfsubject={Don't Act Dumb},
 pdfcreator={Emacs 26.1 (Org mode 9.1.14)},
 pdflang={English}}
\begin{document}

\maketitle
\tableofcontents


\section{God Please Don't Forget These}
\label{sec:org5751d31}
\begin{itemize}
\item Define Algorithm
\item Summations, Logs, Ceiling, Floor
\item Counting Operations
\item Complexities (\(O, \Omega, \Theta\))
\item Divide \& Conquer
\item Greedy Algorithms
\item Dynamic Programming
\end{itemize}

\section{Logs}
\label{sec:org89b7e30}
When it comes to CS, \(\log\) always has a base of 2 unless specified otherwise.
Let's take a look at \(\Theta(n\log n)\).

\subsection{Log Identities}
\label{sec:org6e1880d}
$$\log (x^y) = y\log x$$ \\
$$\log (xy) = \log x + \log y$$ \\
$$\log_b a = \frac{\log_x a}{\log_x b}$$ \\

\subsection{\(\log_{2742} n\) wtf why}
\label{sec:orgd2d8991}
\(\log_{2742} n \leq c\log n \, n \geq k\) \\
\(\frac{\log n}{\log 2742} \leq c\log n\) \\
\(\frac{1}{\log 2742}\log n \leq c\log n \, c=\frac{1}{\log 2742}\) \\

Doing it the limit way: \\
\(\lim_{n\to\infty}\frac{\log_{2742} n}{\log n}\) \\
\(= \lim_{n\to\infty}\frac{\log n}{\log 2742}\frac{1}{\log n}\) \\
\(= \lim_{n\to\infty}\frac{\log n}{\log 2742*\log n}\) \\
\(= \frac{1}{\log2742}\) \\

\section{Complexities}
\label{sec:orgf8be4fb}
\(\Theta(n) + \Theta(1) = \Theta(n+1) = \Theta(n)\) \\
\(\Theta(n) + \Theta(n) = \Theta(n)\) \\
\(\Theta(n) * \Theta(n) * \Theta(n) = \Theta(n^3)\) \\

\section{Dynamic Programming: Optimal Substructure}
\label{sec:orgf0eaf70}
Optimal answers to smaller problems are still applicable. For example, in the coin changing case, it's still better to use the two 6 cent pieces and not the bigger 10 cent piece.
\end{document}

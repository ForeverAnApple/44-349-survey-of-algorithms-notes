% Created 2019-03-01 Fri 12:00
% Intended LaTeX compiler: pdflatex
\documentclass[11pt]{article}
\usepackage[utf8]{inputenc}
\usepackage[T1]{fontenc}
\usepackage{graphicx}
\usepackage{grffile}
\usepackage{longtable}
\usepackage{wrapfig}
\usepackage{rotating}
\usepackage[normalem]{ulem}
\usepackage{amsmath}
\usepackage{textcomp}
\usepackage{amssymb}
\usepackage{capt-of}
\usepackage{hyperref}
\author{davec}
\date{\today}
\title{}
\hypersetup{
 pdfauthor={davec},
 pdftitle={},
 pdfkeywords={},
 pdfsubject={},
 pdfcreator={Emacs 26.1 (Org mode 9.1.14)},
 pdflang={English}}
\begin{document}

\tableofcontents

\#-\textbf{- mode: org -}-

\section{Measuring Execution Time}
\label{sec:org1cc0d1f}
\begin{itemize}
\item It's difficult to measure correctly.
\item Your computer is doing too much.
\item I/O problems.
\item May be too fast, hard to measure correctly.
\item Run out of memory.
\end{itemize}
\subsection{How to Time Correctly}
\label{sec:orgcb4b08e}
\begin{itemize}
\item TIME MULTIPLE RUNS!
\item Statistical significance: 29 or more.
\item Use high resolution timers:
\begin{itemize}
\item Use the smallest measure of time possible.
\end{itemize}
\item Only time the important stuff.
\end{itemize}
\subsection{C++11 Example}
\label{sec:orgd54d1d1}
The following is c++11 only, it uses a high resolution timer rather than tick
from C/C++.
\begin{verbatim}
double time_function()
{
  clock_t start, end, total;
  start = clock();
  for (int i=0; i<ITERS; i++)
    function_to_time();
  end = clock();
  total = end - start;;
  return (total / (float)CLOCKS_PER_SEC) / ITERS;
}
\end{verbatim}

\subsection{Java Example}
\label{sec:org1c70a6c}
\begin{verbatim}
public static float time_method()
{
  long start, end, total;
  start = System.nanoTime();
  // ITERS defined elsewhere
  for (int i=0; i<ITERS; i++)
    method_to_time()
  end = System.nanoTime();
  total = end = start;
  return total / (float) ITERS;
}
\end{verbatim}

\subsection{Python Example}
\label{sec:org016d120}
Refer to python's \href{https://docs.python.org/3.7/library/timeit.html}{Timeit Module (3.7)}.

\section{Math Review}
\label{sec:org8cbfba2}
\subsection{FLOOR: Largest int that is <= x.}
\label{sec:org2622206}
\begin{itemize}
\item floor(3.2) => 3
\item floor(-6.7) => -7
\end{itemize}
\subsection{CEILING: Smallest int that is >= x.}
\label{sec:org5e48513}
\begin{itemize}
\item ceil(3.2) => 4
\item ceil(-6.7) => -6
\end{itemize}
\subsection{LOG}
\label{sec:org760d679}
\begin{itemize}
\item Let b > 1, x > 0
\item Logb(x) = L iff b**L = x
\item Log10(1000) = 3
\item Log2(8) = 3
\item lg = log2
\item lg(16) => 4
\item floor(lg(10)) => 3
\end{itemize}
\subsection{Log identities}
\label{sec:org97aaeb5}
\begin{itemize}
\item logb(1) => 0
\item logb(b) => 1
\item logb(x*y) => logb(x)+logb(y)
\item logb(x/y) => logb(x)-logb(y)
\item logb(x**g) => glogb(x)
\item loga(x) => logb(x)/logb(a) [!]
\end{itemize}
\subsection{ADDITION}
\label{sec:org7b0e463}
\begin{itemize}
\item 1 + 2 + 3 + \ldots{} + n => sum(1, n, i)
\item sum(1, n, 1) => n
\item sum(1, n, i) => n(n+1)/2
\item sum(1, n, i*i) => [n(n+1)(2n+1)]/6
\item sum(1, n, c**i) => (c**(i+1)-1)/(c-1) while c is not 1
\end{itemize}
\end{document}
